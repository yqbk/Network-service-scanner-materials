\documentclass[pdflatex,11pt]{others/aghdpl}
% \documentclass{aghdpl}               % przy kompilacji programem latex
% \documentclass[pdflatex,en]{aghdpl}  % praca w języku angielskim
\usepackage[polish]{babel}
\usepackage[utf8]{inputenc}

% dodatkowe pakiety
\usepackage{enumerate}
\usepackage{listings}

% Default fixed font does not support bold face
\DeclareFixedFont{\ttb}{T1}{txtt}{bx}{n}{8} % for bold
\DeclareFixedFont{\ttm}{T1}{txtt}{m}{n}{8}  % for normal

% Custom colors
\usepackage{color}
\definecolor{deepblue}{rgb}{0,0,0.5}
\definecolor{deepred}{rgb}{0.6,0,0}
\definecolor{deepgreen}{rgb}{0,0.5,0}

\usepackage{listings}
\usepackage{totcount}
\usepackage[nottoc]{tocbibind}

% Python style for highlighting
\newcommand\pythonstyle{\lstset{
language=Python,
basicstyle=\ttm,
otherkeywords={self},             % Add keywords here
keywordstyle=\ttb\color{deepblue},
emph={MyClass,__init__},          % Custom highlighting
emphstyle=\ttb\color{deepred},    % Custom highlighting style
stringstyle=\color{deepgreen},
frame=tb,                         % Any extra options here
showstringspaces=false            % 
}}


% Python environment
\lstnewenvironment{python}[1][]
{
\pythonstyle
\lstset{#1}
}
{}

% Python for external files
\newcommand\pythonexternal[2][]{{
\pythonstyle
\lstinputlisting[#1]{#2}}}

% Python for inline
\newcommand\pythoninline[1]{{\pythonstyle\lstinline!#1!}}

\lstloadlanguages{TeX}

\lstset{
  literate={ą}{{\k{a}}}1
           {ć}{{\'c}}1
           {ę}{{\k{e}}}1
           {ó}{{\'o}}1
           {ń}{{\'n}}1
           {ł}{{\l{}}}1
           {ś}{{\'s}}1
           {ź}{{\'z}}1
           {ż}{{\.z}}1
           {Ą}{{\k{A}}}1
           {Ć}{{\'C}}1
           {Ę}{{\k{E}}}1
           {Ó}{{\'O}}1
           {Ń}{{\'N}}1
           {Ł}{{\L{}}}1
           {Ś}{{\'S}}1
           {Ź}{{\'Z}}1
           {Ż}{{\.Z}}1
}

%---------------------------------------------------------------------------

\author{Jakub Syrek}
\shortauthor{J. Syrek}

\titlePL{Skaner usług sieciowych jako narzędzie dla testów penetracyjnych}
\titleEN{Network services scanner as the penetration testing tool}

\shorttitlePL{Skaner usług sieciowych} % skrócona wersja tytułu jeśli jest bardzo długi
\shorttitleEN{Network services scanner}

\thesistypePL{Praca Inżynierska}
\thesistypeEN{TODOMaster of Science Thesis}

\supervisorPL{dr inż. Michał Turek}
\supervisorEN{Michał Turek ?Ph.D}

\date{2016}

\departmentPL{Katedra Informatyki Stosowanej}
\departmentEN{Department of Automatics}

\facultyPL{WWydział Elektrotechniki, Automatyki, Informatyki i Inżynierii Biomedycznej}
\facultyEN{Faculty of Electrical Engineering, Automatics, Computer Science and Electronics}

\acknowledgements{Serdecznie dziękuję \dots tu ciąg dalszych podziękowań np. dla promotora, żony, sąsiada itp.}



\setlength{\cftsecnumwidth}{10mm}

%---------------------------------------------------------------------------

\begin{document}

%\titlepages

%\tableofcontents
%\clearpage

\chapter{Wprowadzenie}
\label{cha:wprowadzenie}

Szybki rozwój dziedziny nowoczesnych technologii sprawił, że dostęp do
globalnych zasobów sieci Internet stał się możliwy praktycznie w każdym zakątku
ziemi. Już dziś trudno sobie wyobrazić funkcjonowanie oddziałów nowoczesnych
przedsiębiorstw, często umiejscowionych w oddalonych od siebie miastach a nawet i
różnych kontynentach bez wzajemnej komunikacji.

W dzisiejszych czasach kwestia cyber-bezpieczeństwa stała się kwestią priorytetową, nie tylko dla wielkich korporacji i banków lecz także - lub może raczej przede wszystkim - dla małych firm i zwykłych użytkowników komputerów. W celu zapewnienia pewnego pewnego stopnia bezpieczeństwa sieci niezbędne jest wcześniejsze przeanalizowanie istniejących luk i błędów w jej aktualnym stanie zabezpieczeń. W tym celu bardzo często wykorzystuje się skanery portów sieciowych lub też bardziej rozbudowane narzędzia, przedstawiające dodatkowe informacje o badanej sieci, jak na przykład wersja systemu operacyjnego czy też działające usługi sieciowe. Taki oględny skan jest nie tylko wstępem do ataku na daną sieć, lecz może być także świetnym punktem startowym w przypadku, gdy chcemy stworzyć lub wzmocnić zabezpieczenia już istniejące. Aktualnie na rynku znajduję się sporo narzędzi, które można wykorzystać w tym celu, w tej pracy jednak nacisk położony zostanie na zobrazowanie wydajności poszczególnych metod skanowania oraz na przystępną wizualizaję przeprowadzonej analizy z wykorzystaniem interaktywanych diagramów, wykresów i map sieciowych.

%As today’s organizational computer networks are ever evolv- ing and becoming more and more complex, finding potential vulnerabilities and conducting security audits has become a crucial element in securing these networks. The first step in auditing a network is reconnaissance by mapping it to get a comprehensive overview over its structure.


%With the rapid development of the Internet, the information has greatly promoted the economy development and social improvement and it has been the necessary part of people’s life. [1] At the same time, the network security expresses the importance. The increase of relative application and system requirement provides some safeguard for the daily net play. However, the network security of the host cannot obtain the perfect security. It is very common that hacker attacks the online security and brings shadow for network living.


%---------------------------------------------------------------------------

\iffalse
☑ Co? Przedmiot, problem
☑ Jak? (Metoda , krótko)
☐ Dlaczego? Źródła problemu badawczego
☐ Po co? Implikacje, konsekwencje, walory
☐ Co będzie w kolejnych rozdziałach?
\fi

\section{Cele pracy}
\label{sec:celePracy}

Celem poniższej pracy jest zaprojektowanie i stworzenie rozbudowanego narzędzia do prowadzenia testów penetracyjnch na sieciach komputerowych małych i średnich rozmiarów, opartych o skanowanie obecności usług sieciowych z wykorzystaniem jak największej liczby protokołów sieciowych i technik wyszukiwania. Dodatkowo stworzony zostanie podsystem wizualizacji wyników (w postaci diagramów lub map topologii sieci), umożliwiający interaktywne prowadzenie dalszych testów penetracyjnych na wybranych jednostkach. Dodatkowo należy dokonać analizy porównawczej wyników procesów skanowania prowadzonych z przyjęciem różnych priorytetów i kryteriów (czas, wydajność, "anonimowość" skanera, poziom ingerencji).


W celu zaprezentowania wiarygodnych wyników analizy porównawczej przeprowadone zostaną testy na bazie sieci komputerowej niewielkich roziarów liczącej kilka maszyn. Porównianiu poddane zostaną parametry wszystkich omawianych w tej pracy metod i stworzone zostaną dane wizualizujące wyniki testów w celu głębszego zrozumienia różnic między algorytmami.


Źródłem omawianego problemu badawczego jest chęć porównania metod skanowania usług sieciowych pod różnym kątem, w sposób przystępny dla mniej zaawansowanych osób zajmujących się dziedziną bezpieczeństwa sieci komputerowych. Obecnie istnieje spora liczba oprogramowania zajbującego się podobną problematyką, lecz większości z istniejąych rozwiązań brakuje klarownego rozróżnienia pomiędzy poszczególnymi metodami skanowania. 


W zamierzeniu praca ta ma pomóc zrozumieć różnicę między omawianymi w tej prazcy metodami pod względem różnych parametrów, takich jak inwazyjność w badaną sieć czy szybkość działania metody. Na podstawie uzyskanych wyników autor postara się wyłonić metody wyróżniające się od innych pod danym względem.


%---------------------------------------------------------------------------

\section{Zawartość pracy}
\label{sec:zawartoscPracy}

%W rodziale~\ref{cha:pierwszyDokument} przedstawiono podstawowe informacje dotyczące struktury dokumentów w \LaTeX u. Alvis~\cite{Alvis2011} jest językiem

Praca niniejsza składa się z \regtotcounter{chapter} rozdziałów. W drugim pierwszym przedstawiono problematykę związaną ze zwrostem zagrożeń w globalnej sieci internet oraz co ze wzrostem tym się wiąże. Wyjaśniono czym jest i na czym polega prowadzenie testów penetracyjnych. Omówiniono także podstawy działania i budowy protokołów TCP, UDP oraz [datagramów] IP. 


%Rozdziały drugi i trzeci poruszają tematy związane z zagrożeniami przesyłania danych poprzez sieć publiczną oraz opis procesów bezpieczeństwa.
%W rozdziale czwartym znajdują się szczegółowe informacje o technologiach umożliwiających zabezpieczenie danych podczas tranzytu.

\chapter{Podstawy teoretyczne}
\label{cha:teoria}

W rozdziale tym przedstawione zostaną zagadnienia ściśle związane z tematem tej pracy, których znajomość niezbędna jest do zrozumienia kolejnych rozdziałów. 



%---------------------------------------------------------------------------

\section{Wzrost zagrożeń w globalnej sieci Internet}
\label{sec:penTest}

Obecnie sieci IP są w znaczym stopniu zabezpieczone w sposób niewystarczający, co za tym idzie, hakerzy są w stanie wykorzystać istniejące luki w ich infrastrukturze do uzyskania nieautoryzowanego dostępu do wrażliwych danych przedsiębiorstw w celu zakłócenia prawidłowego ich działania %działania serwisów sieciowych. 


Zgodnie z danymi przedstawionymi w raporcie amerykańskiego Federalnego Biura Śledczego, departament  Internet Crime Complaint Center (IC3), \cite{FBI2015} z roku na rok wzrasta ilość strat poniesionych w wyniku cyber-przestępstw ustalając się w roku 2015 na poziomie 1,070.71 milionów dolarów amerykańskich co w porównaniu z rokiem 2001 (17.8 milionów USD) wynosi wzrost o ponad 600 procent. W związku z tak gwałtownym wzrostem strat poniesionych przez przedsiębiorstwa narasta potrzeba zabezpieczenia wrażliwych na ataki infrastuktur informatyczych. 


Instnieją różne sposoby sprawdzenia poziomu bezpieczeństwa instniejącej sieci. Dla przykładu "network vulnerability assessments" badają każdy komponent sieci próbując "determine" szeroką gamę podatności i luk. "Automated vulnerability
scanners" mogą być użyte do rutynowej kontroli sieci i ostatecznie testy penetracyjne testują, czy zabezpieczenia danej sieci mogą być złamane w danym przedziale czasowym.




%---------------------------------------------------------------------------

\section{Czym są testy penetracyjne}
\label{sec:penTest}

Zgodnie z definicją \cite{He2006}, testy penetracyjne sieci są sposobem dla przedsiębiorstw i innych organizacji do odnalezienia luk w zabezpieczneiach na tyle wcześnie, aby zapobiec wykorzystaniu ich przez hakerów do włamania się.


Ważnym pojęciem w tym temacie są "etyczni hakerzy" - osoby zatrudniane do włamania się do sieci a następnie przeprowadzenia licznych ataków przy użyciu technik, które mogą wykorzystać przestępcy internetowi (?).



%---------------------------------------------------------------------------

\section{Czym są testy penetracyjne}
\label{sec:penTest}





%---------------------------------------------------------------------------

\section{Zawartość pracy}
\label{sec:zawartoscPracy}

%Wsasasaas rodziale \ref{cha:wprowadzenie} przedstawiono podstawowe informacje dotyczące struktury dokumentów w \LaTeX u. Alvis~\cite{Alvis2011} jest językiem



\chapter{State of the art}


% nmap 1 , amap 2 , netcat 3 , P0f 4 , XProbe 5 , X-Scan 6 , etc.


% \section{Network Enumeration}

% \subsection{WS_PingProPaek}

% WS_PingProPack wroks as an excellent starting point for any penetra-
% tion test. It has a very user friendly GUI to obtain information on host names and IP
% addresses of the target network. It provides whois, finger, ping, DNS, and SNMP
% information. Additionally WS_PingProPack can be used to search information
% (such as user's full names and e-mail addresses) available through LDAP and
% quickly ping an IP address range or host name.


% \subsection{Sam Spade}

% Sam Spade helps with the discovery phase of penetration testing. Sam
% Spade provides much of the same functionality as WS_Ping Pro-Pack and can per-
% form whois queries, pings, DNS Dig (advanced DNS request), traceroute, finger,
% zone transfers, SMTP mail relay checking and Web site crawling and mirroring.

% \subsection{Rhino9 Pinger}

% After obtainning DNS information about an organisation such as do-
% main names and IP blocks, Pinger is used to find active hosts or targets on the tar-
% get network without being detected. It pings the targets using ICMP ECHO re-
% quests and reply. The pinger sends and ICMP ECHO request and the target sends
% back an ECHO reply.

% \subsection{NetScanTool Pro}

% NetScanTool Pro is a software package with multiple solutions for
% network information discovery, gathering and security testing. It is broadly used
% in network security/administration, Internet forensics and law enforcement.

% %-----------------------------------
% \section{Vulnerability Analysis}

% \subsection{7th Sphere Port Scanner}

% 7th Sphere is an excellent tool for scanning a range of ports on a sin-
% gle host and performing banner grabbing. This information offers vital information
% about the target and the services running on it.

% \subsection{SuperScan}

% Superscan is a fast and powerful connect-based TCP port scanner,
% pinger and hostname resolver. It can connect to any discovered open port using
% user-specified "helper" applications (e.g. Telnet, Web browser, FTP) and assign a
% custom helper application to any port.

% \subsection{Nmap}

% Nmap is considered to be the premier port scanner available as well as
% a reliable OS identification tool, it also serve as a ping sweep utility.


% \subsection{Netcat}

% Netcat is a feature-rich network debugging and exploration tool by
% creating almost any kind of connection using the TCP/IP protocol. It enables user
% to telnet or obtain command line access on different ports, to create back doors, or
% to bypass packet-filtering devices.

% \subsection{What's Running}

% What's running is a banner grabbing program. Once the services run-
% ning on a target host have been identified via port scanning, this program can be
% used to investigate a server to see what HTTP/FTP/POP3/SMTP/NNTP software
% version runs on it. Thus it is easy to understand the associated vulnerabilities.




% %-----------------------------------
% \subsection{Ethereal}

% Ethereal is a network traffic analyser/sniffer for Unix-like operating
% systems. It uses GTK+, a graphical user interface library, and libpcap, a packet cap-
% ture and filtering library.

% \subsection{Nassus}

% Nessus is a remote network security auditor that tests security modules
% in an attempt to discover vulnerabilities. It is made up of two parts: a server, and a
% client. The server/daemon, nessusd, is responsible for the attacks, whereas the cli-
% ent, nessus, interferes with the user through nice X11/GTK+ interface.

% \subsection{Hping2}

% Hping2 is a network tool that can send custom ICMP/UDP/TCP pack-
% ets and display target replies like ping does with ICMP replies. It can be used to
% test firewall rules, perform spoofed port scanning, test net performance using dif-
% ferent protocols, packet size, TOS (type of service), and fragmentation, do path
% MTU discovery, transfer files, trace route, fingerprint remote OSs, audit a TCP/IP
% stack, etc.


% \subsection{Pierwsze próby wytworzenia prostego skanera za pomocą jęzkua python 2.7 i biblioteki pylibcap}

% \section{nmap}

% \section{kali}









% itd.
% \appendix
% \include{dodatekA}
% \include{dodatekB}
% itd.

\bibliographystyle{alpha}
\bibliography{bibliografia}
\begin{thebibliography}{1}

\bibitem{FBI2015}
A.~Diller.
\newblock {\em https://pdf.ic3.gov/2015_IC3Report.pdf }.
\newblock 2015r ?.

% \bibitem{Lam92}
% L.~Lamport.
% \newblock {\em LaTeX system przygotowywania dokumentów}.
% \newblock Wydawnictwo Ariel, Krakow, 1992.

% \bibitem{Alvis2011}
% M.~Szpyrka.
% \newblock {\em {On Line Alvis Manual}}.
% \newblock AGH University of Science and Technology, 2011.cccccc
% \newblock \\\texttt{http://fm.ia.agh.edu.pl/alvis:manual}.

\end{thebibliography}

\end{document}
