\documentclass[pdflatex,11pt]{others/aghdpl}
% \documentclass{aghdpl}               % przy kompilacji programem latex
% \documentclass[pdflatex,en]{aghdpl}  % praca w języku angielskim
\usepackage[polish]{babel}
\usepackage[utf8]{inputenc}

% dodatkowe pakiety
\usepackage{enumerate}
\usepackage{listings}
\lstloadlanguages{TeX}

\lstset{
  literate={ą}{{\k{a}}}1
           {ć}{{\'c}}1
           {ę}{{\k{e}}}1
           {ó}{{\'o}}1
           {ń}{{\'n}}1
           {ł}{{\l{}}}1
           {ś}{{\'s}}1
           {ź}{{\'z}}1
           {ż}{{\.z}}1
           {Ą}{{\k{A}}}1
           {Ć}{{\'C}}1
           {Ę}{{\k{E}}}1
           {Ó}{{\'O}}1
           {Ń}{{\'N}}1
           {Ł}{{\L{}}}1
           {Ś}{{\'S}}1
           {Ź}{{\'Z}}1
           {Ż}{{\.Z}}1
}

%---------------------------------------------------------------------------

\author{Jakub Syrek}
\shortauthor{J. Syrek}

\titlePL{Skaner usług sieciowych jako narzędzie dla testów penetracyjnych}
\titleEN{Network services scanner as the penetration testing tool}

\shorttitlePL{Skaner usług sieciowych} % skrócona wersja tytułu jeśli jest bardzo długi
\shorttitleEN{Network services scanner}

\thesistypePL{Praca Inżynierska}
\thesistypeEN{TODOMaster of Science Thesis}

\supervisorPL{dr inż. Michał Turek}
\supervisorEN{Michał Turek ?Ph.D}

\date{2016}

\departmentPL{Katedra Informatyki Stosowanej}
\departmentEN{Department of Automatics}

\facultyPL{WWydział Elektrotechniki, Automatyki, Informatyki i Inżynierii Biomedycznej}
\facultyEN{Faculty of Electrical Engineering, Automatics, Computer Science and Electronics}

\acknowledgements{Serdecznie dziękuję \dots tu ciąg dalszych podziękowań np. dla promotora, żony, sąsiada itp.}



\setlength{\cftsecnumwidth}{10mm}

%---------------------------------------------------------------------------

\begin{document}

%\titlepages

%\tableofcontents
%\clearpage

\chapter{Wprowadzenie}
\label{cha:wprowadzenie}

W dzisiejszych czasach kwestia cyber-bezpieczeństwa stała się kwestią priorytetową, nie tylko dla wielkich korporacji i banków lecz także - lub może raczej przede wszystkim - dla małych firm i zwykłych użytkowników komputerów. 


%\LaTeX~jest systemem składu umożliwiającym tworzenie dowolnego typu dokumentów (w~szczególności naukowych i technicznych) o wysokiej jakości typograficznej (\cite{Dil00}, \cite{Lam92}). Wysoka jakość składu jest niezależna od rozmiaru dokumentu -- zaczynając od krótkich listów do bardzo grubych książek. \LaTeX~automatyzuje wiele prac związanych ze składaniem dokumentów np.: referencje, cytowania, generowanie spisów (treśli, rysunków, symboli itp.) itd.

%\LaTeX~jest zestawem instrukcji umożliwiających autorom skład i wydruk ich prac na najwyższym poziomie typograficznym. Do formatowania dokumentu \LaTeX~stosuje \TeX a (wymiawamy 'tech' -- greckie litery $\tau$, $\epsilon$, $\chi$). Korzystając z~systemu składu \LaTeX~mamy za zadanie przygotować jedynie tekst źródłowy, cały ciężar składania, formatowania dokumentu przejmuje na siebie system.

%---------------------------------------------------------------------------

\section{Cele pracy}
\label{sec:celePracy}

Celem poniższej pracy jest zapoznanie studentów z systemem \LaTeX~w zakresie umożliwiającym im samodzielne, profesjonalne złożenie pracy dyplomowej w systemie \LaTeX.


%---------------------------------------------------------------------------

\section{Zawartość pracy}
\label{sec:zawartoscPracy}

W rodziale~\ref{cha:pierwszyDokument} przedstawiono podstawowe informacje dotyczące struktury dokumentów w \LaTeX u. Alvis~\cite{Alvis2011} jest językiem

\include{chapters/rozdzial2}

Nessus
nmap



% itd.
% \appendix
% \include{dodatekA}
% \include{dodatekB}
% itd.

\bibliographystyle{alpha}
\bibliography{bibliografia}
%\begin{thebibliography}{1}
%
%\bibitem{Dil00}
%A.~Diller.
%\newblock {\em LaTeX wiersz po wierszu}.
%\newblock Wydawnictwo Helion, Gliwice, 2000.
%
%\bibitem{Lam92}
%L.~Lamport.
%\newblock {\em LaTeX system przygotowywania dokumentów}.
%\newblock Wydawnictwo Ariel, Krakow, 1992.
%
%\bibitem{Alvis2011}
%M.~Szpyrka.
%\newblock {\em {On Line Alvis Manual}}.
%\newblock AGH University of Science and Technology, 2011.cccccc
%\newblock \\\texttt{http://fm.ia.agh.edu.pl/alvis:manual}.
%
%\end{thebibliography}

\end{document}
