\chapter{Wprowadzenie}
\label{cha:wprowadzenie}

Szybki rozwój dziedziny nowoczesnych technologii sprawił, że dostęp do
globalnych zasobów sieci Internet stał się możliwy praktycznie w każdym zakątku
ziemi. Już dziś trudno sobie wyobrazić funkcjonowanie oddziałów nowoczesnych
przedsiębiorstw, często umiejscowionych w oddalonych od siebie miastach a nawet i
różnych kontynentach bez wzajemnej komunikacji.

W dzisiejszych czasach kwestia cyber-bezpieczeństwa stała się kwestią priorytetową, nie tylko dla wielkich korporacji i banków lecz także - lub może raczej przede wszystkim - dla małych firm i zwykłych użytkowników komputerów. W celu zapewnienia pewnego pewnego stopnia bezpieczeństwa sieci niezbędne jest wcześniejsze przeanalizowanie istniejących luk i błędów w jej aktualnym stanie zabezpieczeń. W tym celu bardzo często wykorzystuje się skanery portów sieciowych lub też bardziej rozbudowane narzędzia, przedstawiające dodatkowe informacje o badanej sieci, jak na przykład wersja systemu operacyjnego czy też działające usługi sieciowe. Taki oględny skan jest nie tylko wstępem do ataku na daną sieć, lecz może być także świetnym punktem startowym w przypadku, gdy chcemy stworzyć lub wzmocnić zabezpieczenia już istniejące. Aktualnie na rynku znajduję się sporo narzędzi, które można wykorzystać w tym celu, w tej pracy jednak nacisk położony zostanie na zobrazowanie wydajności poszczególnych metod skanowania oraz na przystępną wizualizaję przeprowadzonej analizy z wykorzystaniem interaktywanych diagramów, wykresów i map sieciowych.


%---------------------------------------------------------------------------

\iffalse
☑ Co? Przedmiot, problem
☑ Jak? (Metoda , krótko)
☐ Dlaczego? Źródła problemu badawczego
☐ Po co? Implikacje, konsekwencje, walory
☐ Co będzie w kolejnych rozdziałach?
\fi

\section{Cele pracy}
\label{sec:celePracy}

Celem poniższej pracy jest zaprojektowanie i stworzenie rozbudowanego narzędzia do prowadzenia testów penetracyjnch na sieciach komputerowych małych i średnich rozmiarów, opartych o skanowanie obecności usług sieciowych z wykorzystaniem jak największej liczby protokołów sieciowych i technik wyszukiwania. Dodatkowo stworzony zostanie podsystem wizualizacji wyników (w postaci diagramów lub map topologii sieci), umożliwiający interaktywne prowadzenie dalszych testów penetracyjnych na wybranych jednostkach. Dodatkowo należy dokonać analizy porównawczej wyników procesów skanowania prowadzonych z przyjęciem różnych priorytetów i kryteriów (czas, wydajność, "anonimowość" skanera, poziom ingerencji).


W celu zaprezentowania wiarygodnych wyników analizy porównawczej przeprowadone zostaną testy na bazie sieci komputerowej niewielkich roziarów liczącej kilka maszyn. Porównianiu poddane zostaną parametry wszystkich omawianych w tej pracy metod i stworzone zostaną dane wizualizujące wyniki testów w celu głębszego zrozumienia różnic między algorytmami.


Źródłem omawianego problemu badawczego jest chęć porównania metod skanowania usług sieciowych pod różnym kątem, w sposób przystępny dla mniej zaawansowanych osób zajmujących się dziedziną bezpieczeństwa sieci komputerowych. Obecnie istnieje spora liczba oprogramowania zajbującego się podobną problematyką, lecz większości z istniejąych rozwiązań brakuje klarownego rozróżnienia pomiędzy poszczególnymi metodami skanowania. 


W zamierzeniu praca ta ma pomóc zrozumieć różnicę między omawianymi w tej prazcy metodami pod względem różnych parametrów, takich jak inwazyjność w badaną sieć czy szybkość działania metody. Na podstawie uzyskanych wyników autor postara się wyłonić metody wyróżniające się od innych pod danym względem.


%---------------------------------------------------------------------------

\section{Zawartość pracy}
\label{sec:zawartoscPracy}

%W rodziale~\ref{cha:pierwszyDokument} przedstawiono podstawowe informacje dotyczące struktury dokumentów w \LaTeX u. Alvis~\cite{Alvis2011} jest językiem

Praca niniejsza składa się z \regtotcounter{chapter} rozdziałów. W drugim pierwszym przedstawiono problematykę związaną ze zwrostem zagrożeń w globalnej sieci internet oraz co ze wzrostem tym się wiąże. Wyjaśniono czym jest i na czym polega prowadzenie testów penetracyjnych. Omówiniono także podstawy działania i budowy protokołów TCP, UDP oraz [datagramów] IP. 


%Rozdziały drugi i trzeci poruszają tematy związane z zagrożeniami przesyłania danych poprzez sieć publiczną oraz opis procesów bezpieczeństwa.
%W rozdziale czwartym znajdują się szczegółowe informacje o technologiach umożliwiających zabezpieczenie danych podczas tranzytu.
