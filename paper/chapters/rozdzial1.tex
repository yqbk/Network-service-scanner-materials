\chapter{Wprowadzenie}
\label{cha:wprowadzenie}

Szybki rozwój dziedziny nowoczesnych technologii sprawił, że dostęp do
globalnych zasobów sieci Internet stał się możliwy praktycznie w każdym zakątku
ziemi. Już dziś trudno sobie wyobrazić funkcjonowanie oddziałów nowoczesnych
przedsiębiorstw, często umiejscowionych w oddalonych od siebie miastach a nawet i
różnych kontynentach bez wzajemnej komunikacji.

W dzisiejszych czasach kwestia cyber-bezpieczeństwa stała się kwestią priorytetową, nie tylko dla wielkich korporacji i banków lecz także - lub może raczej przede wszystkim - dla małych firm i zwykłych użytkowników komputerów. W celu zapewnienia pewnego pewnego stopnia bezpieczeństwa sieci niezbędne jest wcześniejsze zbadanie istniejących luk i błędów w jej aktualnym stanie zabezpieczeń. W tym celu bardzo często wykorzystuje się skanery portów sieciowych lub też bardziej rozbudowane narzędzia, przedstawiające dodatkowe informacje o badanej siecji jak na przykład wersja systemu operacyjnego na komputerach w sieci czy też działające usługi sieciowe. Taki oględny skan jest nie tylko wstępem do ataku na daną sieć, lecz może być także świetnym punktem startowym w przypadku, gdy chcemy stworzyć lub wzmocnić zabezpieczenia już istniejące. Aktualnie na rynku znajduję się sporo narzędzi, które można wykorzystać w tym celu, w tej pracy jednak nacisk położony zostanie na zobrazowanie wydajności poszczególnych metod skanowania sieci oraz na przystępną wizualizaję przeprowadzonej analizy sieci z wykorzystaniem interaktywanych diagramów, wykresów i map sieci.

Konieczne zatem stało się zapewnienie poufności oraz auten-
tyczności, jednym słowem bezpieczeństwa przesyłanych tym medium danych. Mam na
myśli zapobiegnięcie ich podsłuchaniu, nieautoryzowanemu spreparowaniu, przejęciu se-
sji lub też uniemożliwieniu w ogóle komunikacji sieciowej (Atak DoS - Deny Of Service).
Skutki takich ataków podczas komunikacji np. z bankiem, czy też wewnątrz firmy mogą
być paraliżujące na dużą skalę, w konsekwencji - wyjątkowo kosztowne.

%---------------------------------------------------------------------------

\section{Cele pracy}
\label{sec:celePracy}

Celem poniższej pracy jest zaprojektowanie i stworzenie rozbudowanego narzędzia do prowadzenia testów penetracyjnch na sieciach komputerowych małych i średnich rozmiarów, opartych na skanowaniu obecności usług sieciowych z wykorzystaniem jak największej ilości protokołów sieciowych i technik wyszukiwania. Dodatkowo "należy" stworzyć podsystem wizualizacji wyników (w postaci diagramów lub map topologii sieci), umożliwiający interaktywne prowadzenie dalszych testów penetracyjnych na wybranych jednostkach. Dodatkowo należy dokonać analizy porównawczej wyników procesów skanowania prowadzonych z przyjęciem różnych priorytetów i kryteriów (czas, wydajność, "anonimowość" skanera, poziom ingerencji).

Wybrane rozwiązania zostaną zaprojektowane, wdrożone i zweryfikowane w
przykładowej sieci korporacyjnej firmy mającej 6 oddziałów


%---------------------------------------------------------------------------

\section{Zawartość pracy}
\label{sec:zawartoscPracy}

W rodziale~\ref{cha:pierwszyDokument} przedstawiono podstawowe informacje dotyczące struktury dokumentów w \LaTeX u. Alvis~\cite{Alvis2011} jest językiem

Praca niniejsza składa się z 11 rozdziałów. W rozdziale pierwszym
zaprezentowano genezę sieci komputerowych, opis technologii używanych do budowy
sieci rozległych, Internetu i protokołu IP.
Rozdziały drugi i trzeci poruszają tematy związane z zagrożeniami przesyłania danych
poprzez sieć publiczną oraz opis procesów bezpieczeństwa.
W rozdziale czwartym znajdują się szczegółowe informacje o technologiach
umożliwiających zabezpieczenie danych podczas tranzytu.
