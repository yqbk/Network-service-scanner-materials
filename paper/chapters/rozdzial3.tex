\chapter{Implementacja}


\section{pierwsze podejście do implementacji}

\subsection{Pierwsze próby wytworzenia prostego skanera za pomocą jęzkua python 2.7 i biblioteki pylibcap}

Po wnikiliwej analizie istniejących rozwiązań w dziedzinie tworzenia skanerów portów sieciowych, zauważyłem, że większość z nich korzysta z języka pyton [źródło]. Z tego też powodu w początkowej fazie projektu postanowiłem skorzystać z prostej Pthon'owej biblioteki pylibcap i napisać krótki skrypt w języku ptyon w wersji 2.7 w celu sprawdzenia możliwości tej biblioteki. Jednakże już po pierwszych chwilach zorientowałem się, że zaimplementowanie za jej pomocą dużego projektu może okazać się kłopotliwe, ponieważ biblioteka ta jest już dość stara a także niezbyt rozbudowana. Mimo to udało mi się wytworzyć narzędzie mogące przysłużyć się w dalszej części mojego projektu.

\subsubsection{Conn scan}
Na samym pocątku bardzo szubko udało mi się utworzyć narzędzie, które za pomocą najprostszej metody - sprawdzeniu czy podany host odpowiada na próbę połączenia, tak zwany Conn scan - było w stanie uzyskać informacje na temat dostępnych portów na podanej maszynie. 


\begin{python}
def tcp_connect(ip, ports):
    #create a socket
    try:
        #AF_INET -> Internet Protocol v4 addresses, STREAMing socket
        s = socket.socket(socket.AF_INET, socket.SOCK_STREAM)
    except socket.error,err_msg:
      print 'Cannot create a socket'
      sys.exit()

    #checking ports
    for port in ports:
        try:
            #try to connect, if success port is opened
            result = s.connect_ex((ip,port))
            if result == 0:
                print 'port ' + str(port) + ' open'
                pass
            #close scoket and prepare new one
            s.close()
            s = socket.socket(socket.AF_INET, socket.SOCK_STREAM)
        except socket.error:
            pass
\end{python}

