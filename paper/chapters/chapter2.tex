\chapter{Podstawy teoretyczne}
\label{cha:teoria}

W rozdziale tym przedstawione zostaną zagadnienia ściśle związane z tematem tej pracy, których znajomość niezbędna jest do zrozumienia kolejnych rozdziałów. 



%---------------------------------------------------------------------------

\section{Wzrost zagrożeń w globalnej sieci Internet}
\label{sec:penTest}






%---------------------------------------------------------------------------

\section{Czym są testy penetracyjne}
\label{sec:penTest}

Zgodnie z definicją \cite{He2006}, testy penetracyjne sieci są sposobem dla przedsiębiorstw i innych organizacji do odnalezienia luk w zabezpieczneiach na tyle wcześnie, aby zapobiec wykorzystaniu ich przez hakerów do włamania się.



%---------------------------------------------------------------------------

\section{Czym są testy penetracyjne}
\label{sec:penTest}





%---------------------------------------------------------------------------

\section{Zawartość pracy}
\label{sec:zawartoscPracy}

Wsasasaas rodziale \ref{cha:wprowadzenie} przedstawiono podstawowe informacje dotyczące struktury dokumentów w \LaTeX u. Alvis~\cite{Alvis2011} jest językiem


