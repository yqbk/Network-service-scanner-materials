\chapter{Podstawy teoretyczne}
\label{cha:teoria}

W rozdziale tym przedstawione zostaną zagadnienia ściśle związane z tematem tej pracy, których znajomość niezbędna jest do zrozumienia kolejnych rozdziałów. 



%---------------------------------------------------------------------------

\section{Wzrost zagrożeń w globalnej sieci Internet}
\label{sec:penTest}

Obecnie sieci IP są w znaczym stopniu zabezpieczone w sposób niewystarczający, co za tym idzie, hakerzy są w stanie wykorzystać istniejące luki w ich infrastrukturze do uzyskania nieautoryzowanego dostępu do wrażliwych danych przedsiębiorstw w celu zakłócenia prawidłowego ich działania %działania serwisów sieciowych. 


Zgodnie z danymi przedstawionymi w raporcie amerykańskiego Federalnego Biura Śledczego, departament  Internet Crime Complaint Center (IC3), \cite{FBI2015} z roku na rok wzrasta ilość strat poniesionych w wyniku cyber-przestępstw ustalając się w roku 2015 na poziomie 1,070.71 milionów dolarów amerykańskich co w porównaniu z rokiem 2001 (17.8 milionów USD) wynosi wzrost o ponad 600 procent. W związku z tak gwałtownym wzrostem strat poniesionych przez przedsiębiorstwa narasta potrzeba zabezpieczenia wrażliwych na ataki infrastuktur informatyczych. 


Instnieją różne sposoby sprawdzenia poziomu bezpieczeństwa instniejącej sieci. Dla przykładu "network vulnerability assessments" badają każdy komponent sieci próbując "determine" szeroką gamę podatności i luk. "Automated vulnerability
scanners" mogą być użyte do rutynowej kontroli sieci i ostatecznie testy penetracyjne testują, czy zabezpieczenia danej sieci mogą być złamane w danym przedziale czasowym.




%---------------------------------------------------------------------------

\section{Czym są testy penetracyjne}
\label{sec:penTest}

Zgodnie z definicją \cite{He2006}, testy penetracyjne sieci są sposobem dla przedsiębiorstw i innych organizacji do odnalezienia luk w zabezpieczneiach na tyle wcześnie, aby zapobiec wykorzystaniu ich przez hakerów do włamania się.


Ważnym pojęciem w tym temacie są "etyczni hakerzy" - osoby zatrudniane do włamania się do sieci a następnie przeprowadzenia licznych ataków przy użyciu technik, które mogą wykorzystać przestępcy internetowi (?).



%---------------------------------------------------------------------------

\section{Czym są testy penetracyjne}
\label{sec:penTest}





%---------------------------------------------------------------------------

\section{Zawartość pracy}
\label{sec:zawartoscPracy}

%Wsasasaas rodziale \ref{cha:wprowadzenie} przedstawiono podstawowe informacje dotyczące struktury dokumentów w \LaTeX u. Alvis~\cite{Alvis2011} jest językiem


