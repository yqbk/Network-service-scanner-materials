\chapter{State of the art}


% nmap 1 , amap 2 , netcat 3 , P0f 4 , XProbe 5 , X-Scan 6 , etc.


% \section{Network Enumeration}

% \subsection{WS_PingProPaek}

% WS_PingProPack wroks as an excellent starting point for any penetra-
% tion test. It has a very user friendly GUI to obtain information on host names and IP
% addresses of the target network. It provides whois, finger, ping, DNS, and SNMP
% information. Additionally WS_PingProPack can be used to search information
% (such as user's full names and e-mail addresses) available through LDAP and
% quickly ping an IP address range or host name.


% \subsection{Sam Spade}

% Sam Spade helps with the discovery phase of penetration testing. Sam
% Spade provides much of the same functionality as WS_Ping Pro-Pack and can per-
% form whois queries, pings, DNS Dig (advanced DNS request), traceroute, finger,
% zone transfers, SMTP mail relay checking and Web site crawling and mirroring.

% \subsection{Rhino9 Pinger}

% After obtainning DNS information about an organisation such as do-
% main names and IP blocks, Pinger is used to find active hosts or targets on the tar-
% get network without being detected. It pings the targets using ICMP ECHO re-
% quests and reply. The pinger sends and ICMP ECHO request and the target sends
% back an ECHO reply.

% \subsection{NetScanTool Pro}

% NetScanTool Pro is a software package with multiple solutions for
% network information discovery, gathering and security testing. It is broadly used
% in network security/administration, Internet forensics and law enforcement.

% %-----------------------------------
% \section{Vulnerability Analysis}

% \subsection{7th Sphere Port Scanner}

% 7th Sphere is an excellent tool for scanning a range of ports on a sin-
% gle host and performing banner grabbing. This information offers vital information
% about the target and the services running on it.

% \subsection{SuperScan}

% Superscan is a fast and powerful connect-based TCP port scanner,
% pinger and hostname resolver. It can connect to any discovered open port using
% user-specified "helper" applications (e.g. Telnet, Web browser, FTP) and assign a
% custom helper application to any port.

% \subsection{Nmap}

% Nmap is considered to be the premier port scanner available as well as
% a reliable OS identification tool, it also serve as a ping sweep utility.


% \subsection{Netcat}

% Netcat is a feature-rich network debugging and exploration tool by
% creating almost any kind of connection using the TCP/IP protocol. It enables user
% to telnet or obtain command line access on different ports, to create back doors, or
% to bypass packet-filtering devices.

% \subsection{What's Running}

% What's running is a banner grabbing program. Once the services run-
% ning on a target host have been identified via port scanning, this program can be
% used to investigate a server to see what HTTP/FTP/POP3/SMTP/NNTP software
% version runs on it. Thus it is easy to understand the associated vulnerabilities.




% %-----------------------------------
% \subsection{Ethereal}

% Ethereal is a network traffic analyser/sniffer for Unix-like operating
% systems. It uses GTK+, a graphical user interface library, and libpcap, a packet cap-
% ture and filtering library.

% \subsection{Nassus}

% Nessus is a remote network security auditor that tests security modules
% in an attempt to discover vulnerabilities. It is made up of two parts: a server, and a
% client. The server/daemon, nessusd, is responsible for the attacks, whereas the cli-
% ent, nessus, interferes with the user through nice X11/GTK+ interface.

% \subsection{Hping2}

% Hping2 is a network tool that can send custom ICMP/UDP/TCP pack-
% ets and display target replies like ping does with ICMP replies. It can be used to
% test firewall rules, perform spoofed port scanning, test net performance using dif-
% ferent protocols, packet size, TOS (type of service), and fragmentation, do path
% MTU discovery, transfer files, trace route, fingerprint remote OSs, audit a TCP/IP
% stack, etc.


% \subsection{Pierwsze próby wytworzenia prostego skanera za pomocą jęzkua python 2.7 i biblioteki pylibcap}

% \section{nmap}

% \section{kali}




